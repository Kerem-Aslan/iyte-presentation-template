%%%%%%%%%%%%%%%%%%%%%%%%%%%%%%%%%%%%%%%%%
% Beamer Presentation
% LaTeX Template
% Version 2.0 (March 8, 2022)
%
% This template originates from:
% https://www.LaTeXTemplates.com
%
% Author:
% Vel (vel@latextemplates.com)
%
% License:
% CC BY-NC-SA 4.0 (https://creativecommons.org/licenses/by-nc-sa/4.0/)
%
%%%%%%%%%%%%%%%%%%%%%%%%%%%%%%%%%%%%%%%%%

%----------------------------------------------------------------------------------------
%	PACKAGES AND OTHER DOCUMENT CONFIGURATIONS
%----------------------------------------------------------------------------------------

\documentclass[
	12pt, % Set the default font size, options include: 8pt, 9pt, 10pt, 11pt, 12pt, 14pt, 17pt, 20pt
	%t, % Uncomment to vertically align all slide content to the top of the slide, rather than the default centered
	aspectratio=169, % Uncomment to set the aspect ratio to a 16:9 ratio which matches the aspect ratio of 1080p and 4K screens and projectors
]{beamer}

\graphicspath{{Images/}{./}} % Specifies where to look for included images (trailing slash required)

\usepackage{booktabs} % Allows the use of \toprule, \midrule and \bottomrule for better rules in tables
\usepackage{tikz} % For custom header with logo
\usepackage{xcolor} % For custom colors

%----------------------------------------------------------------------------------------
%	SELECT LAYOUT THEME
%----------------------------------------------------------------------------------------

% Beamer comes with a number of default layout themes which change the colors and layouts of slides. Below is a list of all themes available, uncomment each in turn to see what they look like.

%\usetheme{default}
%\usetheme{AnnArbor}
%\usetheme{Antibes}
%\usetheme{Bergen}
%\usetheme{Berkeley}
%\usetheme{Berlin}
%\usetheme{Boadilla}
%\usetheme{CambridgeUS}
%\usetheme{Copenhagen}
%\usetheme{Darmstadt}
%\usetheme{Dresden}
%\usetheme{Frankfurt}
%\usetheme{Goettingen}
%\usetheme{Hannover}
%\usetheme{Ilmenau}
%\usetheme{JuanLesPins}
%\usetheme{Luebeck}
\usetheme{Madrid}
%\usetheme{Malmoe}
%\usetheme{Marburg}
%\usetheme{Montpellier}
%\usetheme{PaloAlto}
%\usetheme{Pittsburgh}
%\usetheme{Rochester}
%\usetheme{Singapore}
%\usetheme{Szeged}
%\usetheme{Warsaw}

%----------------------------------------------------------------------------------------
%	SELECT COLOR THEME
%----------------------------------------------------------------------------------------

% Beamer comes with a number of color themes that can be applied to any layout theme to change its colors. Uncomment each of these in turn to see how they change the colors of your selected layout theme.

% Custom IYTE Color Theme
\definecolor{iyteMaroon}{HTML}{92080F}
\definecolor{iyteMaroonDark}{HTML}{6A0609}
\definecolor{iyteMaroonLight}{HTML}{B52A32}

\setbeamercolor{structure}{fg=iyteMaroon}
\setbeamercolor{palette primary}{bg=iyteMaroon,fg=white}
\setbeamercolor{palette secondary}{bg=iyteMaroonDark,fg=white}
\setbeamercolor{palette tertiary}{bg=iyteMaroonLight,fg=white}
\setbeamercolor{palette quaternary}{bg=iyteMaroon,fg=white}
\setbeamercolor{title}{fg=white,bg=iyteMaroon}
\setbeamercolor{frametitle}{fg=white,bg=iyteMaroon}
\setbeamercolor{block title}{bg=iyteMaroon,fg=white}
\setbeamercolor{block body}{bg=iyteMaroon!10,fg=black}
\setbeamercolor{item}{fg=iyteMaroon}
\setbeamercolor{section in toc}{fg=iyteMaroon}
\setbeamercolor{subsection in toc}{fg=iyteMaroonLight}

%----------------------------------------------------------------------------------------
%	SELECT FONT THEME & FONTS
%----------------------------------------------------------------------------------------

% Beamer comes with several font themes to easily change the fonts used in various parts of the presentation. Review the comments beside each one to decide if you would like to use it. Note that additional options can be specified for several of these font themes, consult the beamer documentation for more information.

\usefonttheme{default} % Typeset using the default sans serif font
%\usefonttheme{serif} % Typeset using the default serif font (make sure a sans font isn't being set as the default font if you use this option!)
%\usefonttheme{structurebold} % Typeset important structure text (titles, headlines, footlines, sidebar, etc) in bold
%\usefonttheme{structureitalicserif} % Typeset important structure text (titles, headlines, footlines, sidebar, etc) in italic serif
%\usefonttheme{structuresmallcapsserif} % Typeset important structure text (titles, headlines, footlines, sidebar, etc) in small caps serif

%------------------------------------------------

%\usepackage{mathptmx} % Use the Times font for serif text
\usepackage{palatino} % Use the Palatino font for serif text

%\usepackage{helvet} % Use the Helvetica font for sans serif text
\usepackage[default]{opensans} % Use the Open Sans font for sans serif text
%\usepackage[default]{FiraSans} % Use the Fira Sans font for sans serif text
%\usepackage[default]{lato} % Use the Lato font for sans serif text

%----------------------------------------------------------------------------------------
%	SELECT INNER THEME
%----------------------------------------------------------------------------------------

% Inner themes change the styling of internal slide elements, for example: bullet points, blocks, bibliography entries, title pages, theorems, etc. Uncomment each theme in turn to see what changes it makes to your presentation.

%\useinnertheme{default}
\useinnertheme{circles}
%\useinnertheme{rectangles}
%\useinnertheme{rounded}
%\useinnertheme{inmargin}

%----------------------------------------------------------------------------------------
%	SELECT OUTER THEME
%----------------------------------------------------------------------------------------

% Outer themes change the overall layout of slides, such as: header and footer lines, sidebars and slide titles. Uncomment each theme in turn to see what changes it makes to your presentation.

%\useoutertheme{default}
%\useoutertheme{infolines}
%\useoutertheme{miniframes}
%\useoutertheme{smoothbars}
%\useoutertheme{sidebar}
%\useoutertheme{split}
%\useoutertheme{shadow}
%\useoutertheme{tree}
%\useoutertheme{smoothtree}

\setbeamertemplate{navigation symbols}{} % Remove navigation symbols

% Custom frametitle with logo on the right
\setbeamertemplate{frametitle}{
    \nointerlineskip
    \begin{beamercolorbox}[wd=\paperwidth,ht=1.0cm,dp=0.2cm]{frametitle}
        \hspace{0.2cm}\raisebox{0.2cm}{\insertframetitle}
        \hfill
        \raisebox{-0.15cm}{\includegraphics[height=1.1cm]{iyte_logo-eng.png}}
        \hspace{0.05cm}
    \end{beamercolorbox}
}

%----------------------------------------------------------------------------------------
%	PRESENTATION INFORMATION, Change accordingly to your presentation
%----------------------------------------------------------------------------------------

\title{Project Title}

\subtitle{Bachelor Thesis Presentation}

% you can change the placeholders, delete or add new information if you like or if you have to or idk, up to you
\author[Kerem Aslan]{\texorpdfstring{Kerem Aslan \\ \medskip Student Number: XXXXXXXXX \\ \medskip Advisor: Dr. Jane Doe}{Kerem Aslan - Advisor: Dr. Jane Doe}}

\institute[IZTECH]{Department of Electrical and Electronics Engineering \\ Izmir Institute of Technology}

\date{\today} % you can change the date, check the commented line below
%\date{8 January 2026}

%----------------------------------------------------------------------------------------

\begin{document}

%----------------------------------------------------------------------------------------
%	TITLE SLIDE
%----------------------------------------------------------------------------------------

\begin{frame}
	\titlepage % Output the title slide, automatically created using the text entered in the PRESENTATION INFORMATION block above
\end{frame}

%----------------------------------------------------------------------------------------
%	TABLE OF CONTENTS SLIDE
%----------------------------------------------------------------------------------------

% The table of contents outputs the sections and subsections that appear in your presentation, specified with the standard \section and \subsection commands. You may either display all sections and subsections on one slide with \tableofcontents, or display each section at a time on subsequent slides with \tableofcontents[pausesections]. The latter is useful if you want to step through each section and mention what you will discuss.

\begin{frame}
	\frametitle{Outline} % Slide title, remove this command for no title
	
	\tableofcontents % Output the table of contents (all sections on one slide)
\end{frame}

% Show table of contents at the beginning of each section with current section highlighted
\AtBeginSection[]{
	\begin{frame}
		\frametitle{Outline}
		\tableofcontents[currentsection]
	\end{frame}
}

%----------------------------------------------------------------------------------------
%	PRESENTATION BODY SLIDES
%----------------------------------------------------------------------------------------

% Include section files
% Introduction and Motivation Section

\section{Introduction and Motivation}

\begin{frame}{Introduction and Motivation}
    \begin{itemize}
        \item Background of the problem
        \item Motivation for the research
        \item Research objectives
    \end{itemize}
\end{frame}

% Theoretical Information Section

\section{Theoretical Information}

\begin{frame}{Theoretical Information}
    \begin{itemize}
        \item Key concepts and definitions
        \item Related work overview
        \item Theoretical foundations
    \end{itemize}
\end{frame}

% Methodology Section

\section{Methodology}

\begin{frame}{Methodology}
    \begin{itemize}
        \item Proposed approach
        \item Algorithm description
        \item Implementation details
    \end{itemize}
\end{frame}

% Experimental Results Section

\section{Experimental Results}

\begin{frame}{Experimental Results}
    \begin{itemize}
        \item Dataset description
        \item Evaluation metrics
        \item Results and analysis
    \end{itemize}
\end{frame}

% Conclusion Section

\section{Conclusion}

\begin{frame}{Conclusion}
    \begin{itemize}
        \item Summary of findings
        \item Main contributions
        \item Future work
    \end{itemize}
\end{frame}


% Include GitHub repository slide
% GitHub Repository Section

\begin{frame}{Source Code}
    \begin{center}
    The source code for this presentation is publicly available at:
    \begin{center}
        {\large \url{https://github.com/Kerem-Aslan/iyte-beamer-presentation/}}
    \end{center}
    \end{center}
\end{frame}


% Closing slide
\begin{frame}
	\begin{center}
		\frametitle{Closing}
		{\Huge Thank you for listening.}
		
		\bigskip\bigskip
		
		{\LARGE Any questions?} 
	\end{center}
\end{frame}

%----------------------------------------------------------------------------------------

% Include References slide
% UNCOMMENT to include references section. You should include your references in the references.bib file, and set the path to the file in references.tex.
%\begin{frame}[allowframebreaks]{References}
    \bibliographystyle{ieeetr}
    \bibliography{references.bib}
    % change the path according to your references file
\end{frame}

\end{document} 